\subsection{coeff}
\label{labcoeff}
\noindent Name: \textbf{coeff}\\
\phantom{aaa}gives the coefficient of degree $n$ of a polynomial\\[0.2cm]
\noindent Library name:\\
\verb|   sollya_obj_t sollya_lib_coeff(sollya_obj_t, sollya_obj_t)|\\[0.2cm]
\noindent Usage: 
\begin{center}
\textbf{coeff}(\emph{f},\emph{n}) : (\textsf{function}, \textsf{integer}) $\rightarrow$ \textsf{constant}\\
\end{center}
Parameters: 
\begin{itemize}
\item \emph{f} is a function (usually a polynomial).
\item \emph{n} is an integer
\end{itemize}
\noindent Description: \begin{itemize}

\item If \emph{f} is a polynomial, \textbf{coeff}(\emph{f}, \emph{n}) returns the coefficient of
   degree \emph{n} in \emph{f}.

\item If \emph{f} is a function that is not a polynomial, \textbf{coeff}(\emph{f}, \emph{n}) returns 0.
\end{itemize}
\noindent Example 1: 
\begin{center}\begin{minipage}{15cm}\begin{Verbatim}[frame=single]
> coeff((1+x)^5,3);
10
\end{Verbatim}
\end{minipage}\end{center}
\noindent Example 2: 
\begin{center}\begin{minipage}{15cm}\begin{Verbatim}[frame=single]
> coeff(sin(x),0);
0
\end{Verbatim}
\end{minipage}\end{center}
See also: \textbf{degree} (\ref{labdegree}), \textbf{roundcoefficients} (\ref{labroundcoefficients}), \textbf{subpoly} (\ref{labsubpoly})
