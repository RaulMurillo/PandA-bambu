\subsection{!$=$}
\label{labneq}
\noindent Name: \textbf{!$=$}\\
\phantom{aaa}negated equality test operator\\[0.2cm]
\noindent Library name:\\
\verb|   sollya_obj_t sollya_lib_cmp_not_equal(sollya_obj_t, sollya_obj_t)|\\[0.2cm]
\noindent Usage: 
\begin{center}
\emph{expr1} \textbf{!$=$} \emph{expr2} : (\textsf{any type}, \textsf{any type}) $\rightarrow$ \textsf{boolean}\\
\end{center}
Parameters: 
\begin{itemize}
\item \emph{expr1} and \emph{expr2} represent expressions
\end{itemize}
\noindent Description: \begin{itemize}

\item The operator \textbf{!$=$} evaluates to true iff its operands \emph{expr1} and
   \emph{expr2} are syntactically unequal and both different from \textbf{error},
   constant expressions that are not constants and that evaluate to two
   different floating-point number with the global precision \textbf{prec} or 
   polynomials that are unequal while automatic expression simplification 
   is activated. The user should be aware of the fact that because of 
   floating-point evaluation, the operator \textbf{!$=$} is not exactly the 
   same as the negation of the mathematical equality. Further, expressions 
   that are polynomials may not be structurally equal when \textbf{!$=$} evaluates 
   to \textbf{false}; in order to obtain purely structural tests, the user should 
   deactivate automatic simplification using \textbf{autosimplify}.
    
   Following the IEEE 754 standard, NaN compares unequal to itself, even though
   this corresponds to a case when \emph{expr1} and \emph{expr2} are syntactically equal
   and different from error. Accordingly, the interval [NaN, NaN] compares
   unequal to itself.
    
   Note that the expressions \textbf{!}(\emph{expr1} \textbf{!$=$} \emph{expr2}) and \emph{expr1} \textbf{$==$} \emph{expr2}
   do not always evaluate to the same boolean value. See \textbf{error} for details.
\end{itemize}
\noindent Example 1: 
\begin{center}\begin{minipage}{15cm}\begin{Verbatim}[frame=single]
> "Hello" != "Hello";
false
> "Hello" != "Salut";
true
> "Hello" != 5;
true
> 5 + x != 5 + x;
false
\end{Verbatim}
\end{minipage}\end{center}
\noindent Example 2: 
\begin{center}\begin{minipage}{15cm}\begin{Verbatim}[frame=single]
> 1 != exp(0);
false
> asin(1) * 2 != pi;
false
> exp(5) != log(4);
true
\end{Verbatim}
\end{minipage}\end{center}
\noindent Example 3: 
\begin{center}\begin{minipage}{15cm}\begin{Verbatim}[frame=single]
> sin(pi/6) != 1/2 * sqrt(3);
true
\end{Verbatim}
\end{minipage}\end{center}
\noindent Example 4: 
\begin{center}\begin{minipage}{15cm}\begin{Verbatim}[frame=single]
> prec = 12;
The precision has been set to 12 bits.
> 16384.1 != 16385.1;
false
\end{Verbatim}
\end{minipage}\end{center}
\noindent Example 5: 
\begin{center}\begin{minipage}{15cm}\begin{Verbatim}[frame=single]
> NaN != NaN;
true
> [NaN,NaN] != [NaN,NaN];
true
> error != error;
false
\end{Verbatim}
\end{minipage}\end{center}
\noindent Example 6: 
\begin{center}\begin{minipage}{15cm}\begin{Verbatim}[frame=single]
> p = x + x^2;
> q = x * (1 + x);
> autosimplify = on;
Automatic pure tree simplification has been activated.
> p != q;
false
> autosimplify = off;
Automatic pure tree simplification has been deactivated.
> p != q;
true
\end{Verbatim}
\end{minipage}\end{center}
See also: \textbf{$==$} (\ref{labequal}), \textbf{$>$} (\ref{labgt}), \textbf{$>=$} (\ref{labge}), \textbf{$<=$} (\ref{lable}), \textbf{$<$} (\ref{lablt}), \textbf{in} (\ref{labin}), \textbf{!} (\ref{labnot}), \textbf{$\&\&$} (\ref{laband}), \textbf{$||$} (\ref{labor}), \textbf{error} (\ref{laberror}), \textbf{prec} (\ref{labprec}), \textbf{autosimplify} (\ref{labautosimplify})
