\subsection{single}
\label{labsingle}
\noindent Names: \textbf{single}, \textbf{SG}\\
\phantom{aaa}rounding to the nearest IEEE 754 single (binary32).\\[0.2cm]
\noindent Library names:\\
\verb|   sollya_obj_t sollya_lib_single(sollya_obj_t)|\\
\verb|   sollya_obj_t sollya_lib_single_obj()|\\
\verb|   int sollya_lib_is_single_obj(sollya_obj_t)|\\
\verb|   sollya_obj_t sollya_lib_build_function_single(sollya_obj_t)|\\
\verb|   #define SOLLYA_SG(x) sollya_lib_build_function_single(x)|\\[0.2cm]
\noindent Description: \begin{itemize}

\item \textbf{single} is both a function and a constant.

\item As a function, it rounds its argument to the nearest IEEE 754 single precision (i.e. IEEE754-2008 binary32) number.
   Subnormal numbers are supported as well as standard numbers: it is the real
   rounding described in the standard.

\item As a constant, it symbolizes the single precision format. It is used in 
   contexts when a precision format is necessary, e.g. in the commands 
   \textbf{round} and \textbf{roundcoefficients}. In is not supported for \textbf{implementpoly}.
   See the corresponding help pages for examples.
\end{itemize}
\noindent Example 1: 
\begin{center}\begin{minipage}{15cm}\begin{Verbatim}[frame=single]
> display=binary!;
> SG(0.1);
1.10011001100110011001101_2 * 2^(-4)
> SG(4.17);
1.000010101110000101001_2 * 2^(2)
> SG(1.011_2 * 2^(-1073));
0
\end{Verbatim}
\end{minipage}\end{center}
See also: \textbf{halfprecision} (\ref{labhalfprecision}), \textbf{double} (\ref{labdouble}), \textbf{doubleextended} (\ref{labdoubleextended}), \textbf{doubledouble} (\ref{labdoubledouble}), \textbf{quad} (\ref{labquad}), \textbf{tripledouble} (\ref{labtripledouble}), \textbf{roundcoefficients} (\ref{labroundcoefficients}), \textbf{implementpoly} (\ref{labimplementpoly}), \textbf{round} (\ref{labround}), \textbf{printsingle} (\ref{labprintsingle})
