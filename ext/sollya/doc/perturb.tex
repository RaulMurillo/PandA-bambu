\subsection{perturb}
\label{labperturb}
\noindent Name: \textbf{perturb}\\
\phantom{aaa}indicates random perturbation of sampling points for \textbf{externalplot}\\[0.2cm]
\noindent Library names:\\
\verb|   sollya_obj_t sollya_lib_perturb()|\\
\verb|   int sollya_lib_is_perturb(sollya_obj_t)|\\[0.2cm]
\noindent Usage: 
\begin{center}
\textbf{perturb} : \textsf{perturb}\\
\end{center}
\noindent Description: \begin{itemize}

\item The use of \textbf{perturb} in the command \textbf{externalplot} enables the addition
   of some random noise around each sampling point in \textbf{externalplot}.
    
   See \textbf{externalplot} for details.
\end{itemize}
\noindent Example 1: 
\begin{center}\begin{minipage}{15cm}\begin{Verbatim}[frame=single]
> bashexecute("gcc -fPIC -c externalplotexample.c");
> bashexecute("gcc -shared -o externalplotexample externalplotexample.o -lgmp -l
mpfr");
> externalplot("./externalplotexample",relative,exp(x),[-1/2;1/2],12,perturb);
\end{Verbatim}
\end{minipage}\end{center}
See also: \textbf{externalplot} (\ref{labexternalplot}), \textbf{absolute} (\ref{lababsolute}), \textbf{relative} (\ref{labrelative}), \textbf{bashexecute} (\ref{labbashexecute})
