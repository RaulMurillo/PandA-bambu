\subsection{accurateinfnorm}
\label{labaccurateinfnorm}
\noindent Name: \textbf{accurateinfnorm}\\
\phantom{aaa}computes a faithful rounding of the infinity norm of a function \\[0.2cm]
\noindent Usage: 
\begin{center}
\textbf{accurateinfnorm}(\emph{function},\emph{range},\emph{constant}) : (\textsf{function}, \textsf{range}, \textsf{constant}) $\rightarrow$ \textsf{constant}\\
\textbf{accurateinfnorm}(\emph{function},\emph{range},\emph{constant},\emph{exclusion range 1},...,\emph{exclusion range n}) : (\textsf{function}, \textsf{range}, \textsf{constant}, \textsf{range}, ..., \textsf{range}) $\rightarrow$ \textsf{constant}\\
\end{center}
Parameters: 
\begin{itemize}
\item \emph{function} represents the function whose infinity norm is to be computed
\item \emph{range} represents the infinity norm is to be considered on
\item \emph{constant} represents the number of bits in the significant of the result
\item \emph{exclusion range 1} through \emph{exclusion range n} represent ranges to be excluded 
\end{itemize}
\noindent Description: \begin{itemize}

\item The command \textbf{accurateinfnorm} computes an upper bound to the infinity norm of
   function \emph{function} in \emph{range}. This upper bound is the least
   floating-point number greater than the value of the infinity norm that
   lies in the set of dyadic floating point numbers having \emph{constant}
   significant mantissa bits. This means the value \textbf{accurateinfnorm} evaluates to
   is at the time an upper bound and a faithful rounding to \emph{constant}
   bits of the infinity norm of function \emph{function} on range \emph{range}.
    
   If given, the fourth and further arguments of the command \textbf{accurateinfnorm},
   \emph{exclusion range 1} through \emph{exclusion range n} the infinity norm of
   the function \emph{function} is not to be considered on.

\item The command \textbf{accurateinfnorm} is now considered DEPRECATED in \sollya.
   Users should be aware about the fact that the algorithm behind
   \textbf{accurateinfnorm} is highly inefficient and that other, better suited
   algorithms, such as \textbf{supnorm}, are available inside \sollya. As a
   matter of fact, while \textbf{accurateinfnorm} is maintained for compatibility reasons
   with legacy \sollya codes, users are advised to avoid using \textbf{accurateinfnorm}
   in new \sollya scripts and to replace it, where possible, by the
   \textbf{supnorm} command.
\end{itemize}
\noindent Example 1: 
\begin{center}\begin{minipage}{15cm}\begin{Verbatim}[frame=single]
> p = remez(exp(x), 5, [-1;1]);
> accurateinfnorm(p - exp(x), [-1;1], 20);
4.52055246569216251373291015625e-5
> accurateinfnorm(p - exp(x), [-1;1], 30);
4.5205513970358879305422306060791015625e-5
> accurateinfnorm(p - exp(x), [-1;1], 40);
4.520551396713923253400935209356248378753662109375e-5
\end{Verbatim}
\end{minipage}\end{center}
\noindent Example 2: 
\begin{center}\begin{minipage}{15cm}\begin{Verbatim}[frame=single]
> p = remez(exp(x), 5, [-1;1]);
> midpointmode = on!;
> infnorm(p - exp(x), [-1;1]);
0.45205~5/7~e-4
> accurateinfnorm(p - exp(x), [-1;1], 40);
4.520551396713923253400935209356248378753662109375e-5
\end{Verbatim}
\end{minipage}\end{center}
See also: \textbf{infnorm} (\ref{labinfnorm}), \textbf{dirtyinfnorm} (\ref{labdirtyinfnorm}), \textbf{supnorm} (\ref{labsupnorm}), \textbf{checkinfnorm} (\ref{labcheckinfnorm}), \textbf{remez} (\ref{labremez}), \textbf{diam} (\ref{labdiam})
